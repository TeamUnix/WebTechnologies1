\documentclass[10pt,a4paper]{report}
\usepackage[latin1]{inputenc}
\usepackage[english]{babel}
\usepackage{amsmath}
\usepackage{amsfonts}
\usepackage{amssymb}
\usepackage{amsthm}
\usepackage{graphicx}
\usepackage{fancyhdr}
\usepackage{lastpage}
\usepackage{multirow}
\usepackage{float}
\usepackage{wrapfig}

\setcounter{secnumdepth}{5}
\setcounter{tocdepth}{5}


\oddsidemargin  -0.5cm
\evensidemargin 0.0cm
\textwidth      17.25cm
\headheight     1.0cm
\headsep		0.7cm
\topmargin      -0.5cm
\textheight		22.0cm

\pagestyle{fancy}
%identation in the beginning of each paragraph and 2 lines between paragraph
\setlength{\parindent}{1cm}

\lhead{Energy Hub}
\chead{Web1}
\rhead{\thepage\ of \pageref{LastPage}}
\lfoot{Dennis Madsen\\Theis Christensen\\Paulo Fontes}
\cfoot{Team3}
\rfoot{\today}
\renewcommand{\headrulewidth}{0.4pt}
\renewcommand{\footrulewidth}{0.4pt}



%%%%%%%%%%%%%%%% LSTLISTING %%%%%%%%%%%%%%%%%%
%Include and define  c code
\usepackage{listings}
\usepackage{color}
\usepackage{textcomp}
\definecolor{listinggray}{gray}{0.9}
\definecolor{lbcolor}{rgb}{0.9,0.9,0.9}

\usepackage{xcolor}         % Text farbig markieren                     

\definecolor{colKeys}{rgb}{0,0,1} 
\definecolor{colIdentifier}{rgb}{1,0,0} 
\definecolor{colComments}{rgb}{0,0.7,0.4} 
\definecolor{colString}{rgb}{0,0.5,0} 

\lstset{%
    language=HTML,
    float=hbp,% 
    basicstyle=\ttfamily\small, % 
    identifierstyle=\color{colIdentifier}, % 
    keywordstyle=\color{colKeys}, % 
    stringstyle=\color{colString}, % 
    commentstyle=\color{colComments}, % 
    columns=flexible, % 
    tabsize=2, % 
    frame=single, % 
    extendedchars=true, % 
    showspaces=false, % 
    showstringspaces=false, % 
    numbers=left, % 
    numberstyle=\tiny, % 
    breaklines=true, % 
    backgroundcolor=\color{lbcolor}, % 
    breakautoindent=true, % 
    captionpos=b% 
} 

\lstdefinelanguage{CSS} 
{morekeywords={color,background,margin,padding,font,weight,display,position,top,left,right,bottom,list,style,border,size,white,space,min,width},
sensitive=false, 
morecomment=[l]{//}, 
morecomment=[s]{/*}{*/}, 
morestring=[b]", 
} 

\begin{document}
\section{Small Screens Optimisation}

\subsection{Constrains and Limitations}
On small screens horizontal scroll bar is not an option, since it breaks all the dynamics of the user experience, the processor is slower the memory is limited so loading large file sizes is not a plus.\\
For small screen some limitations should be considered and applied to the design:\\
- One column layout only, this will avoid the horizontal scroll bar.\\
- HTML optimised using efficient tags and attributes.\\
- CSS optimised with efficient styles proprieties.\\
- Minimise the amount of decorative images, or if needed, use of pre loaders (scripts that only show the page when fully loaded) so the user experience is not affected.\\
- Write good alternative text (attribute 'alt')  for images, so in case the image is not needed an alternative text could be shown for example navigation icons.\\
- Avoid effects that need mouse or keyboard events.\\
- Avoid the use of Javascript on layout design since some mobile browsers may be set to block this scripts.

\subsubsection{Screen Size Proposal}
Smaller screens will always be the tablets PCs or the ones that fits on pockets like smartphones. Keeping this in mind the most used screen sizes are:\\
- 320x480 Screen Size ( eg. iPhone 3G )\\
- 1024x600 Screen Size ( eg. Samsung Galaxy )\\
- 1024x768 Screen Size ( eg. iPad 1)\\
This is are the most common used smartphones and tablets, but the same device can have different resolution for example the case of iPhone3G (320x480) and iPhone4 ();
\\
\textbf{Make some research on the most used devices / mobile browsers / screen sizes.}
\\

\subsubsection{Avoiding Images}
All images are loaded to the memory, being this very limited on smartphones and tablet computers, its necessary to "hide" some images so the browser don't load them, this have to be done without compromising the user experience.\\
If images are needed they should be scaled down on a image editor (e.g. PhotoShop), this way the size of the image ( quota ) will get smaller so it will take less memory space and less time to be loaded and shown to the user.\\
User should always be informed when the page is not completely loaded, this can be done using a preload script (script that hides the content of a page showing a loading bar until the page finish loading). This kind of scripts avoids a bad navigation experience to the user.\\

\subsubsection{Re-Design}
To make a page suitable for a small screen, the page have to be re-designed, in this case, only the CSS will be changed keeping the HTML tags intact.\\ This method of web development have advantages since the HTML only have the contents of the page and not any layout characteristics the space needed on the server is going to be smaller and the development is faster since only the CSS is changed to fit the content to a smaller screen.\\
When the webpage is open, a CSS ( Cascade Style Sheet ) is loaded defining the layout for the entire page. After loading and designing the layout as a normal screen, the bowser will see the device characteristics, and change the already design layout to the correspondent one. This way we can change only the properties that are different for different screen sizes.\\

\subsection{Page Layout}
For a long time tables were used to make the webpages layout, this was made by nesting HTML table tags to create for example multi column pages, separate headers, main and footers, etc..\\ As solution, HTML should be used to the intellectual content of the page, while CSS determine how to represent the content to the user. This method of design improves the time of development since theres no more nested tables the HTML tags become more understandable, so create and redesigned a webpage becomes more easy and less time consuming, improves search engines rankings and files sizes diminished noticeable.\\

\subsubsection{Handling Orientation}  % CSS / JAVASCRIPT
CSS is able to get information about almost all newer smartphone devices. But still not able to distinguish from an iPhone3G and an iPhone 4, this will be a problem since both are the same device with different screen sizes.\\

\subsubsection{Changing layout}	    % CSS / PHOTOSHOP
Everything has a percentage....
CSS must have the size of the screen otherwise it screws up for iPad.. keep tablets loading the normal screen...
\subsubsection{Scaling Down}		    % CSS / PHOTOSHOP
Scaling everything down on photoshop...
Talk about image compression, GIF, PNG, JPEG...
No scale down with CSS and why...

\subsubsection{User Experience}	    % CSS

\section{References}
CSS Cookbook O'Reilly 2.Edition Christopher Schmitt.

\end{document}