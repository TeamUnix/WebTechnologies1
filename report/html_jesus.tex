\documentclass[10pt,a4paper]{report}
\usepackage[latin1]{inputenc}
\usepackage[english]{babel}
\usepackage{amsmath}
\usepackage{amsfonts}
\usepackage{amssymb}
\usepackage{amsthm}
\usepackage{graphicx}
\usepackage{fancyhdr}
\usepackage{lastpage}
\usepackage{multirow}
\usepackage{float}
\usepackage{wrapfig}

\setcounter{secnumdepth}{5}
\setcounter{tocdepth}{5}


\oddsidemargin  -0.5cm
\evensidemargin 0.0cm
\textwidth      17.25cm
\headheight     1.0cm
\headsep		0.7cm
\topmargin      -0.5cm
\textheight		22.0cm

\pagestyle{fancy}
%identation in the beginning of each paragraph and 2 lines between paragraph
\setlength{\parindent}{1cm}

\lhead{Energy Hub}
\chead{Web1}
\rhead{\thepage\ of \pageref{LastPage}}
\lfoot{Dennis Madsen\\Theis Christensen\\Paulo Fontes}
\cfoot{Team3}
\rfoot{\today}
\renewcommand{\headrulewidth}{0.4pt}
\renewcommand{\footrulewidth}{0.4pt}



%%%%%%%%%%%%%%%% LSTLISTING %%%%%%%%%%%%%%%%%%
%Include and define  c code
\usepackage{listings}
\usepackage{color}
\usepackage{textcomp}
\definecolor{listinggray}{gray}{0.9}
\definecolor{lbcolor}{rgb}{0.9,0.9,0.9}

\usepackage{xcolor}         % Text farbig markieren                     

\definecolor{colKeys}{rgb}{0,0,1} 
\definecolor{colIdentifier}{rgb}{1,0,0} 
\definecolor{colComments}{rgb}{0,0.7,0.4} 
\definecolor{colString}{rgb}{0,0.5,0} 

\lstset{%
    language=HTML,
    float=hbp,% 
    basicstyle=\ttfamily\small, % 
    identifierstyle=\color{colIdentifier}, % 
    keywordstyle=\color{colKeys}, % 
    stringstyle=\color{colString}, % 
    commentstyle=\color{colComments}, % 
    columns=flexible, % 
    tabsize=2, % 
    frame=single, % 
    extendedchars=true, % 
    showspaces=false, % 
    showstringspaces=false, % 
    numbers=left, % 
    numberstyle=\tiny, % 
    breaklines=true, % 
    backgroundcolor=\color{lbcolor}, % 
    breakautoindent=true, % 
    captionpos=b% 
} 

\lstdefinelanguage{CSS} 
{morekeywords={color,background,margin,padding,font,weight,display,position,top,left,right,bottom,list,style,border,size,white,space,min,width},
sensitive=false, 
morecomment=[l]{//}, 
morecomment=[s]{/*}{*/}, 
morestring=[b]", 
} 

\begin{document}
\section{Small Screens Optimisation}
\subsection{Constrains}
On small screens there is no horizontal scroll, the processor is slower the memory is limited so download large amounts of images is not a plus.\\
For small design some limitations should be applied to the design:\\
- One column layout only\\
- HTML optimised using efficient tags and attributes.\\
- CSS optimised with efficient styles.\\
- Minimize the amount of decorative images, or if needed, use of pre loaders (scripts that only show the page when fully loaded) so the user experience is not affected.\\
- Write good 'alt', alternative text for images.\\
- Avoid effects that need mouse clicks or keyboard.\\
\subsection{Screen Size Proposal}
Smaller screens will always be the ones that fits on pockets or used as books readers. Keeping this in mind the most used screen sizes are:\\
- 320x480 Screen Size ( eg. iPhone 3G )\\
- 1024x600 Screen Size ( eg. Samsung Galaxy )\\
- 1024x768 Screen Size ( eg. iPad 1)\\
\subsection{Avoiding Images}
All images are loaded to the memory, being this very limited on a smartphone and tablet computers, its necessary to take some images out without compromising the user experience.\\
If images are needed they should be scaled down on a image editor (e.g. PhotoShop), this way the size of the image will get smaller so it will take less memory space and less time to be loaded to the user.\\
User should always be informed when the page is not completely loaded, this can be done using a preload script (script that hides the content of a page showing a loading bar until the page finish loading).\\ 
\end{document}